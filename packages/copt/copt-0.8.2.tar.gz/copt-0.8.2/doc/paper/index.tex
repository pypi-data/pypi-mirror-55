\documentclass[twoside,11pt]{article}

\usepackage[nohyperref, preprint]{jmlr2e}

\usepackage{amsmath}
% \usepackage{amsthm}
\usepackage{amssymb}
\usepackage{empheq}
\usepackage{xcolor, color, colortbl}
\usepackage{mdframed}
\usepackage{pifont}
\newcommand{\cmark}{\ding{51}}%
\newcommand{\xmark}{\ding{55}}%
\usepackage{enumitem}
% For figures
\usepackage{graphicx} % more modern


\newcommand{\blue}{\color{blue}}

\definecolor{mydarkblue}{rgb}{0,0.08,0.45}
\usepackage[colorlinks=true,
    linkcolor=mydarkblue,
    citecolor=mydarkblue,
    filecolor=mydarkblue,
    urlcolor=mydarkblue,
    pdfview=FitH]{hyperref}

\graphicspath{{./figures/}}


\jmlrheading{1}{2019}{1-48}{4/00}{10/00}{X}{Authors}

% Short headings should be running head and authors last names

\ShortHeadings{C-OPT: Composite Optimization in Python}{Pedregosa}
\firstpageno{1}


\begin{document}

\title{C-OPT: Composite Optimization in Python}
\author{\name Fabian Pedregosa \email pedregosa@google.com \\
       \addr Google Research\\
}
\editor{}


\maketitle


\begin{abstract}
\emph{copt} is a Python library integrating a wide range of classical optimization algorithm for medium-scale problems. By packaging a wide array of optimization algorithms into a consistent API, this package focuses on brining optimization algorithms to practitioners. Emphasis is on robustness, doocumentation, performance and API consistency. It has minimal dependencies and is distributed under the Apache-2.0 license, encouraging its use in both academic and commercial settings.
\end{abstract}

\begin{keywords}
  optimization, python
\end{keywords}

\section{Introduction}

{\blue Big environment Python for scientific computing. \newcommand{\blue}{\color{blue}}
}

\section{Project Vision}

\paragraph{Code quality.}

\paragraph{Bare-bones design and API.}

\paragraph{Documentation.}

\paragraph{Apache license.}


\section{Underlying Technologies}

\citep{virtanen2019scipy}

\citep{pedregosa2011scikit}

\section{Computational}

\bibliography{biblio}


\end{document}
