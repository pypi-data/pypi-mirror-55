\documentclass[tikz, multi=tikzpicture, margin=5mm]{standalone}

\usepackage[sfdefault, condensed]{roboto}
\usepackage[T1]{fontenc}
\usepackage[utf8]{inputenc}

\usetikzlibrary{decorations}
\usetikzlibrary{decorations.text}
\usetikzlibrary{decorations.markings}
\usetikzlibrary{decorations.pathreplacing}

\usetikzlibrary{arrows,arrows.meta,positioning,shapes,mindmap,trees,backgrounds,
  fit,calc,intersections}

\tikzstyle{io} = [trapezium, trapezium left angle=70, trapezium right angle=110, minimum width=3.2cm, minimum height=1cm, text centered, draw=black, fill=DivColorblindSafe0, text width=2.8cm]

\tikzstyle{process} = [rectangle, minimum width=3cm, minimum height=1cm, text centered, draw=black, fill=DivColorblindSafe2!50, text width=2.8cm]

\tikzstyle{inoutpar} = [rectangle, text centered, fill=white, draw=black, minimum width=2cm, minimum height=2em]

% activity
\tikzstyle{act} = [rectangle, rounded corners=2mm, text centered, draw=black, fill=DivColorblindSafe2!50, text width=3cm, minimum height=2em, font=\bfseries]

\tikzstyle{start} = [circle, fill=black, scale=1.5]


\tikzstyle{comment} = [rectangle, dotted, minimum width=4cm, minimum height=1cm, text centered, draw=black, fill=gray!30, text width=3.8cm, scale=0.8]

\tikzstyle{decision} = [diamond, minimum width=3cm, minimum height=1cm, text centered, draw=black, fill=green!30]

% \def\myarrow{{Latex[scale=1.5]}}
\def\myarrow{{Straight Barb[scale=2]}}
\tikzstyle{arrow} = [-\myarrow]
\pgfarrowsdeclaredouble{<}{>}{Straight Barb[scale=2]}{}
\pgfarrowsdeclaredouble{<<}{>>}{Square[open, scale=2]}{Straight Barb[scale=2]}

\pgfarrowsdeclarealias{b}{b}{Square[open, scale=2]}{Square[open, scale=2]}
\pgfarrowsdeclaredouble{b<}{>b}{Square[open, scale=2]}{Straight Barb[scale=2]}


\tikzstyle{hp} = [trapezium, trapezium left angle=70, trapezium right angle=110, fill=DivColorblindSafe1, minimum height=2em, text width=2.8cm, text centered]

\tikzstyle{dot} = [circle, scale=0.3, fill=black]

\tikzstyle{decision} = [diamond, draw, text badly centered, inner sep=3pt]


% custom shape, following https://stuff.mit.edu/afs/athena/contrib/tex-contrib/beamer/pgf-1.01/doc/generic/pgf/version-for-tex4ht/en/pgfmanualse25.html
\makeatletter
\pgfdeclareshape{frametitle}{
  \inheritsavedanchors[from=rectangle] % this is nearly a rectangle
  \inheritanchorborder[from=rectangle]
  \inheritanchor[from=rectangle]{center}
  \inheritanchor[from=rectangle]{north}
  \inheritanchor[from=rectangle]{south}
  \inheritanchor[from=rectangle]{west}
  \inheritanchor[from=rectangle]{east}
  \inheritanchor[from=rectangle]{north west}

  % ... and possibly more
  \backgroundpath{% this is new
    % store upper right in xa/ya and lower right in xb/yb
    \northeast \pgf@xa=\pgf@x \pgf@ya=\pgf@y
    \southwest \pgf@xb=\pgf@x \pgf@yb=\pgf@y
    % extend box to the right
    \pgf@xa=\pgf@xa \advance\pgf@xa by+4pt
    % compute corner of ``flipped page''
    \pgf@xc=\pgf@xa \advance\pgf@xc by-6pt % this should be a parameter
    \pgf@yc=\pgf@yb \advance\pgf@yc by+6pt
    % construct main path
    \pgfpathmoveto{\pgfpoint{\pgf@xa}{\pgf@ya}} % upper right
    \pgfpathlineto{\pgfpoint{\pgf@xa}{\pgf@yc}} % corner start
    \pgfpathlineto{\pgfpoint{\pgf@xc}{\pgf@yb}} % corner end
    \pgfpathlineto{\pgfpoint{\pgf@xb}{\pgf@yb}} % lower left
    \pgfpathlineto{\pgfpoint{\pgf@xb}{\pgf@ya}} % upper left
    \pgfpathclose
  }
}\makeatother


% freely after https://tex.stackexchange.com/a/47281
\tikzset{stop/.style={draw, circle, anchor=center, scale=1.4, line width=1pt, path picture={%
      \node [anchor=center, circle, fill=black, scale=1] {};
}}}

% freely after https://tex.stackexchange.com/a/73166
\tikzset{
  outpin/.style={
    draw, scale=0.82, append after command={
      (\tikzlastnode.north west) edge (\tikzlastnode.south)
      (\tikzlastnode.north east) edge (\tikzlastnode.south)
    }
  }
}

\tikzstyle{pinlbl} = [inner sep=6pt, font=\footnotesize]
\tikzstyle{lbl} = [pinlbl]

\tikzset{
  mybackground2/.style n args={3}{execute at end picture={
      \begin{scope}[on background layer]
        \draw[black, thick] #1 rectangle #2;
        \node[draw, yshift=0pt, frametitle, anchor=north west, fill=black!20, thick, inner sep=3pt]
        at #1 {\scriptsize #3};
      \end{scope}
    }
  }
}

\tikzstyle{mybackground} = [mybackground2={(current bounding box.north west)}{(current bounding box.south east)}{#1}]

% http://colorbrewer2.org/#type=diverging&scheme=PuOr&n=3
\definecolor{DivPhotocopySafe0}{RGB}{247,247,247}
\definecolor{DivPhotocopySafe1}{RGB}{241,163,64}
\definecolor{DivPhotocopySafe2}{RGB}{153,142,195}

% http://colorbrewer2.org/#type=diverging&scheme=BrBG&n=3
\definecolor{DivColorblindSafe0}{RGB}{245,245,245}
\definecolor{DivColorblindSafe1}{RGB}{216,179,101}
\definecolor{DivColorblindSafe2}{RGB}{90,180,172}


\begin{document}

\begin{tikzpicture}[
    node distance = 8mm,
    mybackground2={
      ($(current bounding box.north west)+(12mm, -6mm)$)
    }{
      ($(current bounding box.south east)+(5mm, 5mm)$)
    }{
      \textbf{act} map\_feature
    },
    % show background grid, show background rectangle,
    act/.append style/={text width=32mm},
    inoutpar/.append style/={text width=32mm},
  ]

  % ANCHOR : INPUT
  \node [inoutpar, text width=30mm] (in-obs) {
    Observations with in-features
  };

  \begin{scope}[below of=in-obs, yshift=-10mm, local bounding box=mapping]

    \newlength\boxwidth\setlength{\boxwidth}{28mm}

    \node [draw, inner sep=0pt, fill=white] (mapping input) {
      $~~\vert~~\vert~~\vert~~$
    };

    \node [act, below=of mapping input] (map) {greedily replace with out-features};
    \draw [->b](mapping input) -- (map.north)
    node [above right, pinlbl] {in-features};
    
    \node [draw, inner sep=0pt, below=of map, fill=white] (mapping output) {
      $~~\vert~~\vert~~\vert~~$
    };
    \draw [b->] (map.south) node [below right, pinlbl] {out-features} -- (mapping output);
    
    \begin{pgfonlayer}{background}
      \draw[dashed, thick] (mapping input.center -| -\boxwidth,0)
           [rounded corners=5mm] rectangle (mapping output -| \boxwidth, 0);
    \end{pgfonlayer}

    % text
    \node [above left, lbl, text width=35mm, align=right] at (mapping input.west) {
      observations\\\scriptsize[in-features]
    };

    \node [below left, lbl, text width=35mm, align=right] at (mapping output.west) {
      observations\\\scriptsize[out-features]
    };

  \end{scope}

  \draw [->] (in-obs) -- (mapping input);


  % INPUT mapping matrix
  \node [inoutpar, left=18mm of map, text width=25mm] (mm) {
    sorted \mbox{mapping matrix}
  };
  \draw [->b] (mm) -- (map);

  % OUTPUT
  \node [inoutpar, below=1cm of mapping output] (out-obs) {
    Observations with out-features
  };
  \draw [->] (mapping output) -- (out-obs);
\end{tikzpicture}


\begin{tikzpicture}[
    node distance = 8mm,
    mybackground2={($(current bounding box.north west)+(10mm, -5mm)$)}%
    {($(current bounding box.south east)+(0, 5mm)$)}%
    {\textbf{act} map\_feature\_smin},
    % show background grid, show background rectangle,
    act/.append style/={text width=32mm},
    inoutpar/.append style/={text width=32mm},
  ]

  % ANCHOR : INPUT: X0
  \node [inoutpar] (in-obs) {Observations with in-features};

  % diags split
  \node [decision, below=of in-obs] (diags split) {};
  \draw [->] (in-obs) -- (diags split);

  % (map) assign out-featurs to obs
  \node [act, below=8mm of diags split] (map) {
    map\_feature
  };
  \draw [->b] (diags split) -- (map.north)
  node [above right, pinlbl, text width=2cm] {
    observations with in-features
  };

  % decide if done
  \node [decision, below=15mm of map] (decide) {};
  \draw [b->] (map) -- (decide);

  \node [act, left=of map, text width=50mm] (update) {
    remove first weakly supported out-feature from mapping matrix
  };
  \draw [->b] (decide) node [lbl, above left, text width=40mm, align=right] {
    [$\exists$ weakly supported out-feature (i.e., with support < s\textsubscript{min})]
  } -| (update);
  \node [decision] (mm split) at (update |- diags split) {};
  \draw [b->] (update) -- (mm split);
  \draw [->] (mm split) -- (diags split);

  \node [inoutpar, below=of decide] (out-obs) {Observations with out-features};
  \draw [->] (decide) node [lbl, below right, text width=35mm] {
    [otherwise]
  } -- (out-obs);

  % INPUT mapping matrix
  \node [inoutpar, left=10mm of mm split, text width=25mm] (mm) {
    sorted \mbox{mapping matrix}
  };
  \draw [->] (mm) -- (mm split);

\end{tikzpicture}


\end{document}
